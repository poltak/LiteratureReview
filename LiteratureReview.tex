\documentclass[a4paper,11pt]{article}

\usepackage[paper=a4paper,left=30mm,width=150mm,top=25mm,bottom=25mm]{geometry} 
\usepackage{fullpage}

\usepackage{amsmath}
\usepackage{amssymb}
\usepackage{graphicx}
\usepackage{gensymb}
\usepackage{enumitem}
\usepackage{setspace} % This is used in the title page
\usepackage{graphicx} % This is used to load the crest in the title page
\usepackage{subcaption}
\usepackage{afterpage}
\usepackage[hidelinks]{hyperref}
\usepackage[sec\\\
ion]{placeins}

\usepackage[hidelinks]{hyperref}

\usepackage{booktabs}

\usepackage[parfill]{parskip}

\graphicspath{{./Figures/}}
\long\def\/*#1*/{}

\usepackage{titlesec}

\begin{document}

% Set up a title page
\thispagestyle{empty} % no page number on very first page
% Use roman numerals for page numbers initially
\renewcommand{\thepage}{\roman{page}}

\begin{spacing}{1.5}
\begin{center}
{\Large \bfseries
Faculty of Information Technology\\
Monash University}

\vspace*{30mm}

\includegraphics[width=5cm]{MonashCrest}

\vspace*{15mm}

{\large \bfseries
Literature Review --- Semester 2, 2014
}

\vspace*{10mm}

{\LARGE \bfseries
Isolated Region Spatial Query
}

\vspace*{20mm}

{\large \bfseries
Jian Loong Liew 22545727

\vspace*{20mm}

Supervisor: Associate Professor David Taniar
}

\end{center}
\end{spacing}

\newpage

\tableofcontents

\newpage
\setcounter{page}{1}
\renewcommand{\thepage}{\arabic{page}}

\section{Introduction} 

The \textit{Isolated Region} spatial query is a distinctive spatial query
because of its requirements. It aims to achieve the very nature of being near
a crowd yet not too far that it is isolated. It is visually easy to determine
an isolated region given any number of objects bound within a region. However,
it mathematically solving this in the terms of both the Euclidean and Spatial
distance is challenging. The input of this query will be a set of static data
points. $P=\{p_1,....,p_n\}$. Each of these points will denote objects of
interests (OOIs). The output of this query will be a region, $R$. This region
will be the coined \textit{Isolated Region}.

In order to propose a query processing solution to this query, the various
nearest neighbour queries with their query processing solutions will be
explored. This literature review attempts to categorize the various spatial
query into different categories and identify the ones of interest with regards
to solving the spatial query processing scenario at hand. The tradional
reverse nearest neighbour query only returns objects, however the inverse
version of these queries will return a region. These queries are also called
the reverse queries.

Below is a graphical representation of the various spatial queries and the
various algorithms proposed by authors to solve the scenario at hand. In order
to begin attempting to propose a query processing solution for the problem at
hand the understanding of the work done by previous authors in order to solve
similar problems needs to be understood.

\/*
\subsection{Overview of Spatial Queries}

This diagram is based on deriving information from ~\cite{taniar2013taxonomy}.

\section{Nearest Neighbour}

The nearest neighbour problem is considered to be one of the best known
problems in computer science.

\subsection{Range Euclidean Restriction}

~\cite{papadias2003query}. 

\subsection{Aggregate Nearest Neighbour Query}

Aggregate nearest neighbour queries returns the object that minimizes an
aggregate distance function with respect to a set of query
points~\cite{yiu2005aggregate}. A scenario for ANN is as follows: Assumming
for example, $n$ users at location $(q_1,....,q_2)$. An ANN query outputs the
facility that minimises the sum of distances that the users have to travel in
order to meet there~\cite{papadias2005aggregate}. Thus, the input for an ANN
query will be the set of static data points given to it.

\subsection{Constrained Nearest Neighbour}

~\cite{ferhatosmanoglu2001constrained}.

\subsection{Reverse Nearest Neighbour}

A Reverse Nearest Neighbour (RNN) search is a method to retrieve all objects
that consider the query point as the nearest neighbour. An example would be
like when a marketing application in which the issue is to determine the
business impact of opening an outlet of Company \textit{A} at a given
location. A simple task is to determine the segment of \textit{A}'s customer
who would be likely to use this new facility~\cite{korn2000influence}. Korn et
al (2000) broken this down into two cases which are static and dynamic cases
and also formalised a novel notion of influence based on reverse nearest
neighbour queries and its variants. Influence sets based on reverse nearest
neighbour (RNN) queries seem to caputre the notion of influence based on the
example stated by them.

Cheema et al (2011), suggested a more generic concept called \textbf{influence
zone} and showed that the influence zone can be used to efficiently compute
the influence sets. This influence zone has various applications in location
based services and decision support systems. This is because the influence
zone may be used for market analysis as well as targeted
marketing~\cite{cheema2011influence}. This concept is more generic compared to
the notation of influence set proposed by Korn et al (2000). An RNN query is
independant of interest, and

\section{Reverse Region Query}

The input of this are objects and the output of this query is a region. 

\section{Group Nearest Neighbour} 

This was proposed by Papadias et al~\cite{papadias2004group}. It is considered
to be a novel form of Nearest Neighbour search. The GNN query is different
from the traditional \textit{k}nn query which only specifies a single query
point, the GNN query has multiple query points. The input of this problem
consist of  a set of $P=\{p_1,...,p_n\}$ of static data points in multi
dimensional space and a group of query points. $Q=\{q_1,...,q_n\}$. The ouput
of this contains the $k>1$ data points~\cite{papadias2004group}.

An example scenario for the GNN would would be to select a meeting place from
all available meeting places. For example, for three executive directors who
are located in three different places, to meet at the closest meeting
place~\cite{taniar2013taxonomy}.

This is considered to be an expensive problem by definition because of the
number of data points as well as query points it needs to process as it is
considerably more complex than the traditional \textit{k}nn queries. The
reason for this complexity is mainly due to two reasons~\cite{li2005two}. The
first is because there are multiple query points being specified which
requires more distance computation and the other is because the fact that the
query point can be distributed within the data space in arbitary ways,
creating a large search region.


\subsection{Multiple Query Method (MQM)}

This algorithm utilises the threshold algorithm where it performs incremental
NN queries for each point in $Q$ and combines their
results~\cite{papadias2004group}. The MQM retrieves the NN for every point in
query set $Q$, it sometimes accesses the same tree nodes for different query
points and this causes its cost to increase fast with the query set
cardinality.

\subsection{Single Point Method}

\subsection{Minimum Bounding Method (MBM)}

\subsection{Group Cloest Pair Method}

\subsection{Two Ellipse-based Pruning Method}

Li et al (2005) suggested a distance pruning method using an ellipse. The
pruning method is ``If a point or an MBR is far away enough with respect to
the two points we choose as the approximate ellipse, they cannot be in the
final answer.'' This method is compared to the SPM method and the MBM method.
The authors claim that this method works more efficiently than both SPM and
MBM because the ellipse used in the methods are less distance computational
during the search and can prune unqualified nodes more efficiently. IT is
noted that the two ellipse-based pruning method can be used in both the depth-
first and best-first travel paradigms.

\section{Probablistic Group Nearest Query}

~\cite{lian2008probabilistic}.

\section{Group Nearest Group}

Group Nearest Group (GNG) query can be defined as a query which finds one data
point $p$ from a data point set $D$ such that the total distance from $p$ to
the points in a query point set $Q$ is minimal~\cite{deng2012group}. This is
regarded as the generic version of the GNN query. When $k$ = 1, a GNG query is
reduced to a GNN query. A GNG query can also be called as a k-median
clustering in operations search which is a partition based clustering problem
with group data points into $k$ which is a given number of clusters based on
an optimization objective function.

The scenario of a GNG query is as follows. A security service provider plans
to set up several new branches to serve several business districts which can
be represented by a set of land marks, such as well-known buildings. The max
number of branches is usually on constrainted by factors such as business
cost. Under this constraint the provider wishes to select branch locationsfrom
many choices such that the reponse time to security alarms can be minimised
that is the average distance between these landmarks to the nearest branch is
minimal.

The GNG algorithm has its useful because it will find a meeting point from a
set of groups, however it requires 2 set of inputs which are the

The GNG query processing starts by taking an inital subsect of $W_mi$. 

Query points with different weights are also useful in many
situations~\cite{deng2012group}.

\subsection{Exhaustive Hierarchical Combination Algorithm (EHC)}

\subsection{Subset Hierarchical Replacement Algorithm (SHR)}

\subsection{Comparison of query processing solutions}

\subsection{Zone of Influence}
 

\section{Optimum Region} 

An optimum region query is as follows: 

Another spatial query of interest is the \textit{Optimum Region}. This problem
appears in Euclidean space. Optimum region is defined as a region which
includes all the points which can cover the maximum number of objects in the
finite set as the center of a radius \textit{r}~\cite{Xuan2012}. The objective
of the optimum region search is to find a region where the center of a circle
with a fixed radius can be located to cover most objects of interest in the
given set. An example of an optimum region is as follows:


\begin{itemize}      
  \item Build a hospital in a community and let most
resident reach to the hospital within 20 minutes or 15 km.      
  \item A company plans to establish a new Wi-Fi base station, the coverage range of
which is 20 meters, to cover most Wi-Fi   devices in the company.
\end{itemize}

The input for the optimum region are a set of object of interest and the
output is the optimum region.

\subsection{Circle Partition and Arcs Superposition (CPAS)}
This algorithm relies on the polar coordinates of the system.

\section{Skyline Query}


\section{Isolated Region}

\section{R-Trees} In order to efficiently solve the query processing in the
multi dimensional splace, an efficient indexing system has been proposed by
various authors. An example would be the \textit{R-trees}~\cite{guttman1984r}.
This has then been further developed and enhanced by various authors. For
example, a variation of the \textit{R-trees} is the $R^+ -trees$ that avoids
overlapping rectangles in intermediate nodes of the tree that is
introduced.~\cite{beckmann1990r}.

\subsection{Rectangle Handling}
Table from R+ tree goes here.

\section{TPR\* Tree}
~\cite{tao2003tpr}.

\section{Voronoi Diagram}

\section{Comparison}

The table below list the various spatial queries based on their input, output as well as the various algorithms used.

\begin{tabular}{llr}
\hline
\multicolumn{2}{c}{Item} \\
\cline{1-2}
Name    & Input \& Output & Algorithm \\
\hline
Aggregate Nearest Neighbour      & per gram    & 13.65      \\
Reverse Nearest Neighbour          & each        & 0.01       \\
Gnu       & stuffed     & 92.50      \\
Group Nearest Group (GNG)       & stuffed     & 33.33      \\
Optimum Region & frozen      & Circle Partition and Arcs Superposition (CPAS)       \\
\hline
\end{tabular}



\clearpage
\bibliographystyle{plain}
\bibliography{LiteratureReview}


\end{document}